\documentclass[12pt]{article}
\usepackage{graphicx}
\usepackage{geometry}
\usepackage[table]{xcolor}
\usepackage{titlesec}
\usepackage{fancyhdr}
\setlength{\headheight}{15pt}
\usepackage{array}
\geometry{margin=1in}

% Define custom colors
\definecolor{IonBlue}{RGB}{0,70,140}
\definecolor{IonGray}{RGB}{240,240,240}
\definecolor{IonAccent}{RGB}{200,50,50}

% Section formatting
\titleformat{\section}
  {\color{IonBlue}\normalfont\Large\bfseries}
  {\color{IonAccent}\thesection}{1em}{}

% Header and footer
\pagestyle{fancy}
\fancyhf{}
\fancyhead[L]{\color{IonBlue}HardHaQ '25 Ion Trap Challenge}
\fancyhead[R]{\color{IonAccent}\thepage}
\fancyfoot[C]{\color{IonBlue}Team Submission}

% Title
\title{\textcolor{IonBlue}{HardHaQ '25 Trapped Ion Problem Set Submission}}
\author{\textcolor{IonAccent}{Team Name:} \textit{Hard Nanos} \\ 
        \textcolor{IonAccent}{Members:} \textit{Nikhil, Rebanta, Lucas}}
\date{\ November 22 2025}

\begin{document}
\maketitle

\section{Introduction}
Briefly state the objective of your design (e.g., maximize trap depth while minimizing RF power).
Mention the starting point (the provided COMSOL file) and your overall strategy.

\section{Design Choices}
\begin{itemize}
    \item \textbf{Geometry modifications:} [Describe changes to rod spacing, rod length, endcap placement, or custom shapes.]
    \item \textbf{Parameter tuning:} [List RF voltage, DC endcap voltages, or other editable parameters you adjusted.]
    \item \textbf{Rationale:} [Explain why you made these changes — symmetry improvement, deeper potential well, reduced offset, etc.]
\end{itemize}

\section{Trap Metrics Results}
Insert your exported Trap Metrics table here. Example format:

\rowcolors{2}{IonGray}{white}
\begin{center}
\begin{tabular}{|>{\columncolor{IonGray}}l|c|l|}
\hline
\textbf{Metric} & \textbf{Value} & \textbf{Notes} \\
\hline
depth\_eV & [ ] & Trap depth (higher = stronger confinement) \\
minU\_eV & [ ] & Minimum effective potential \\
maxU\_eV & [ ] & Maximum effective potential \\
trap\_x, y, z & [ ] & Coordinates of trap minimum \\
offset\_mm & [ ] & Distance from geometric center \\
P\_est\_mW & [ ] & Estimated RF power \\
\hline
\end{tabular}
\end{center}

Highlight improvements compared to the default configuration. If possible, show before vs. after values.

\section{Visual Evidence}
Include at least one screenshot of your trap geometry and potential distribution. Example:

\begin{figure}[htbp]
    \centering
    \includegraphics[width=0.7\textwidth]{trap_geometry.png}
    \caption{\textcolor{IonBlue}{Modified trap geometry with reduced offset.}}
\end{figure}

\section{Analysis \& Discussion}
Explain how your design affected confinement quality:
\begin{itemize}
    \item Did trap depth increase?
    \item Was the ion better centered?
    \item Did RF power efficiency improve?
\end{itemize}

Discuss trade-offs (e.g., deeper trap but higher power, symmetry vs. complexity).
Note any unexpected artifacts or limitations.

\section{Conclusion}
Summarize why your design is effective.
State the main improvement achieved (e.g., “Our design reduced offset by 40\% while maintaining comparable depth”).

\section{Optional Extensions}
If you explored unconventional geometries, parameter sweeps, or anisotropic traps, describe them briefly.
Mention any future directions or open questions.

\section*{Deliverables Checklist}
\begin{itemize}
    \item Exported Trap Metrics table (.txt file)
    \item Screenshot(s) of geometry and potential distribution
    \item Modified COMSOL file (.mph)
    \item Written summary (this document)
\end{itemize}

\end{document}
