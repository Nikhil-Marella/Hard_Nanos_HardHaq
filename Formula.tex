\documentclass[12pt]{article}
\usepackage{graphicx}
\usepackage{caption}
\usepackage{geometry}
\usepackage{enumitem}
\usepackage{hyperref}
\usepackage[table]{xcolor}
\usepackage{titlesec}
\usepackage{fancyhdr}
\setlength{\headheight}{15pt}
\usepackage{array}
\usepackage{amsmath}
\geometry{margin=1in}
\usepackage{pgfplotstable}
\pgfplotsset{compat=1.18}
\usepackage[font=small,labelfont=bf]{caption}


% Define custom colors
\definecolor{IonBlue}{RGB}{0,70,140}
\definecolor{IonGray}{RGB}{240,240,240}
\definecolor{IonAccent}{RGB}{200,50,50}

% Section formatting
\titleformat{\section}
  {\color{IonBlue}\normalfont\Large\bfseries}
  {\color{IonAccent}\thesection}{1em}{}

% Header and footer
\pagestyle{fancy}
\fancyhf{}
\fancyhead[L]{\color{IonBlue}HardHaQ '25 Ion Trap Challenge}
\fancyhead[R]{\color{IonAccent}\thepage}
\fancyfoot[C]{\color{IonBlue}Team Submission}

% Title
\title{\textcolor{IonBlue}{HardHaQ '25 Trapped Ion Problem Set Submission}}
\author{\textcolor{IonAccent}{Team Name:} \textit{Hard Nanos} \\ 
        \textcolor{IonAccent}{Members:} \textit{Nikhil, Rebanta, Lucas}}
\date{November 22, 2025}

\begin{document}
\maketitle

\bigskip

% The conventions and definitions were previously listed here. They are
% now placed inline immediately after the equations where each symbol is
% first used so readers see definitions in context.

\section{Mathematical Framework}
\subsection{Shorthand useful relations}
The Mathieu parameter (radial) is
\begin{equation}
  q \;=\; \frac{2 Q V}{m r_0^2 \Omega^2}.
\end{equation}
% Definitions introduced at first use
Here $Q$ is the ion charge (C), $m$ is the ion mass (kg), $V$ is the RF peak amplitude (V), $\Omega$ is the RF angular frequency (rad\,s$^{-1}$), and $r_0$ is the characteristic trap radius (m).
The secular frequency (radial secular motion) in the small-$q$ limit is approximated by
\begin{equation}
  \omega_{\mathrm{sec}} \;\approx\; \frac{q}{2\sqrt{2}}\,\Omega
  \;=\; \frac{Q V}{\sqrt{2}\, m r_0^2 \Omega}.
\end{equation}

\subsection{Pseudopotential depth ({$U_{\mathrm{depth}}$})}
\subsubsection{Governing formula}
The time-averaged pseudopotential (in 1D radial coordinate $r$ for an ideal quadrupole) is
\[
\Psi(r) \;=\; \frac{Q^2}{4 m \Omega^2}\,|\mathbf{E}_{\rm rf}(r)|^2.
\]
% Definition: pseudopotential
\par
\noindent Here $\Psi(\mathbf r)$ is the time-averaged pseudopotential (J).
For the ideal quadrupole field amplitude $|\mathbf{E}_{\rm rf}|\!\sim\! V r / r_0^2$, this gives a quadratic pseudopotential:
\[
\Psi(r) \;=\; \kappa\,\frac{Q^2 V^2}{4 m \Omega^2 r_0^4}\; r^2,
\]
so a characteristic \emph{trap depth} (energy scale between center and effective boundary $r\!\sim\!r_0$) is commonly approximated as
\begin{equation}\label{eq:Udepth}
  U_{\rm depth} \;\approx\; \kappa\,\frac{Q^2 V^2}{4 m \Omega^2 r_0^2}.
\end{equation}
\par
\noindent Here $U_{\rm depth}$ denotes the trap depth (J) and $\kappa$ is a dimensionless geometry factor (order unity).
An alternative useful form in terms of the Mathieu parameter $q$ is
\begin{equation}\label{eq:Udepth_q}
  U_{\rm depth} \;=\; \kappa\,\frac{m \Omega^2 r_0^2}{16}\; q^2.
\end{equation}

\subsubsection{How each variable affects depth}
\begin{itemize}
  \item \textbf{RF amplitude $V$:} $U_{\rm depth}\propto V^2$. \emph{Increasing $V$ increases depth quadratically.}
  \item \textbf{Drive frequency $\Omega$:} $U_{\rm depth}\propto 1/\Omega^2$. \emph{Increasing $\Omega$ reduces depth (inverse square).}
  \item \textbf{Trap size $r_0$:} $U_{\rm depth}\propto 1/r_0^2$. \emph{Increasing $r_0$ decreases depth (inverse square).}
  \item \textbf{Ion properties $Q,m$:} $U_{\rm depth}\propto Q^2/m$. Higher charge increases depth; heavier mass decreases depth.
  \item \textbf{Geometry $\kappa$:} improves depth linearly with $\kappa$ (ideal electrodes have $\kappa\!\approx\!1$).
\end{itemize}

\subsubsection{Practical remarks and trade-offs}
\begin{itemize}
  \item Increasing $V$ increases $q$. Large $q$ invalidates the pseudopotential approximation and increases micromotion.
\end{itemize}

\subsection{Center offset ({$x_0$}) — static displacement and micromotion}
\subsubsection{Governing formula}
If a stray DC electric field $E_{\rm dc}$ exists at the trap center, it exerts a force $Q E_{\rm dc}$ on the ion. Balance this against the secular restoring force $m \omega_{\mathrm{sec}}^2 x$ to obtain the static displacement (offset) $x_0$:
\begin{equation}\label{eq:x0}
  x_0 \;=\; \frac{Q E_{\rm dc}}{m \omega_{\mathrm{sec}}^2}.
\end{equation}
\par
\noindent where $E_{\mathrm{dc}}$ is the stray DC electric field (V/m) producing the static offset.
Using the small-$q$ expression for $\omega_{\mathrm{sec}}$,
\[
\omega_{\mathrm{sec}} \approx \frac{Q V}{\sqrt{2}\, m r_0^2 \Omega},
\]
we can express $x_0$ in terms of trap parameters:
\begin{equation}\label{eq:x0_expanded}
  x_0 \;\approx\; \frac{Q E_{\rm dc}}{m}\left(\frac{\sqrt{2}\, m r_0^2 \Omega}{Q V}\right)^{\!2}
  \;=\; \frac{2 m r_0^4 \Omega^2}{Q V^2}\;E_{\rm dc}.
\end{equation}

\subsubsection{Excess micromotion amplitude (leading order)}
For small $q$ the amplitude of \emph{excess} micromotion driven at RF frequency is approximately proportional to $q$ times the static offset:
\begin{equation}
  x_{\rm mm} \;\approx\; \frac{q}{2}\,x_0.
\end{equation}
Kinetic energy associated with excess micromotion scales as $\sim \tfrac{1}{2} m (\Omega x_{\rm mm})^2$.

\subsubsection{How each variable affects center offset}
\begin{itemize}
  \item \textbf{Stray field $E_{\rm dc}$:} $x_0\propto E_{\rm dc}$ (linear). More stray field $\Rightarrow$ larger offset.
  \item \textbf{Secular stiffness $\omega_{\rm sec}$:} $x_0\propto 1/\omega_{\rm sec}^2$. Stronger confinement (higher $\omega_{\rm sec}$) $\Rightarrow$ smaller $x_0$.
  \item \textbf{RF amplitude $V$:} Increasing $V$ increases $\omega_{\rm sec}$ (for fixed $\Omega$ and $r_0$) and thus reduces $x_0$; but increasing $V$ also increases $q$ and so multiplies offset into micromotion amplitude.
  \item \textbf{Drive frequency $\Omega$:} Appears in $\omega_{\mathrm{sec}}$ inversely; higher $\Omega$ (for fixed $V$) tends to reduce $x_0$ via $\omega_{\mathrm{sec}}$ formula in practice.
  \item \textbf{Trap size $r_0$:} Larger $r_0$ reduces secular frequency (for fixed $V,\Omega$) so $x_0$ grows with $r_0$.
\end{itemize}

\subsubsection{Minimization strategies}
\begin{itemize}
  \item \textbf{Increase secular stiffness:} Raise $V$ or reduce $r_0$ (within other constraints) to reduce $x_0$.
  \item \textbf{Balance trade-offs:} Increasing $V$ reduces $x_0$ but increases $q$ (and thus micromotion amplitude $x_{\rm mm}\propto q x_0$). Optimize to minimize residual micromotion energy.
\end{itemize}

\subsection{RF power and power-per-depth (\texorpdfstring{$P/U_{\mathrm{depth}}$}{P/U_depth})}
\subsubsection{Model for RF power}
The RF power is modelled by the following equation extracted from the COMSOL file:
\begin{equation}\label{eq:Pdiss}
  P_{\mathrm{est}} \;=\; 10^{3}\left(\omega_{\mathrm{sec}}\big(\varepsilon\pi L\big)\frac{V_{\mathrm{rf}}}{\sqrt{2}}\right)^{2}
\end{equation}
Here $V_{\mathrm{rf}}$ is the RF peak amplitude. In practice, $P_{\mathrm{est}}$ also depends on the effective electrode capacitance and series resistance of the RF drive and electrode structure (but, we assume it is negligible).

\subsubsection{Power per depth}
Combine \eqref{eq:Pdiss} with the depth formula \eqref{eq:Udepth} which scales as $U_{\rm depth}\propto V^2$. Eliminating $V^2$ yields:
\begin{equation}\label{eq:PoverU_raw}
  \frac{P_{\mathrm{est}}}{U_{\rm depth}}
  \;\propto\; \frac{10^{3}\left(\omega_{\mathrm{sec}}\big(\varepsilon\pi L\big)\frac{V_{\mathrm{rf}}}{\sqrt{2}}\right)^{2}}{\;\kappa\,\frac{m \Omega^2 r_0^2}{16}\; q^2\;}  \Rightarrow ...
  \Rightarrow \frac{P_{\mathrm{est}}}{U_{\rm depth}} \propto \frac{10^{3}(\varepsilon\pi L)^{2}  V_{\mathrm{rf}}^{2}}{\kappa m r_{0}^{2}}.
\end{equation}
where the constant prefactor depends on the precise definitions used for $U_{\rm depth}$ and $P_{\rm est}$; the displayed scaling is the relevant design dependence.

\subsubsection{How each variable affects $P/U_{\rm depth}$}
\begin{itemize}
  \item \textbf{Trap size $r_0$:} $P/U_{\rm depth}\propto r_0^2$ (scales with trap area/size) — larger traps cost more power per depth.
  \item \textbf{Ion properties:} $P/U_{\rm depth}\propto m/Q^2$ — heavier ions increase power per depth; higher ionic charge reduces it.
  \item \textbf{RF voltage $V_{\rm rf}$:} $P/U_{\mathrm{depth}}\propto V_{\rm rf}^{2}$;  
      smaller $V_{\rm rf}$ reduces power cost.
  \item \textbf{Electrode length $L$:} $P/U_{\mathrm{depth}}\propto L^{2}$;  
      smaller $L$ reduces power cost if the quadrupole field quality is preserved.

\end{itemize}

\subsubsection{Minimization of $P/U_{\rm depth}$}
\begin{enumerate}
  \item \textbf{Choose moderate $\Omega$:} Avoid excessively high drive frequencies unless required for stability/performance.
  \item \textbf{Reduce $r_0$ where acceptable:} Smaller traps require less RF voltage for a given depth.
\end{enumerate}

\end{document}